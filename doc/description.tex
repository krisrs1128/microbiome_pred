\documentclass{article}
\usepackage[autostyle]{csquotes}  
\include{preamble}

\title{Abundance Modeling for Exploratory Micribiome Analysis}
\author{Kris Sankaran}

\begin{document}
\maketitle

\begin{displayquote}
The success of a theory is best judged from its ability to predict
  in new contexts.
\end{displayquote}

This comment appeared in D.R. Cox's (otherwise critical) discussion of
\cite{breiman2001statistical}, and it will serve as the point of departure in
our effort to better understand the microbiome using machine learning models.
The idea is that microbiome researchers are trying to get their hands on a
theory\footnote{While I'm quoting things, here's another relevant one: ``There
  is nothing so practical as a good theory'' -- Kurt Lewin} of the bacterial
ecology and the associated medical consequences. For example, can the theory
predict what happens to the abundance of Bacteroides in the short and long run,
when a patient undergoes a colon cleanout?

Building prediction models for bacterial abundances has never been that
intrinsically interesting (in the same way it is in finance or online search,
say, though maybe things will change if personalized microbiome abundance
becomes common practice). We argue however that, while the $\hat{y}_{i}$'s
generated by such a model might not be particularly significant in and of
themselves, there are two ways these $\hat{y}_{i}$'s can be used to guide real
theory-building,

\begin{itemize}
\item Data Reduction: An algorithm's prediction surface thought of as a kind of
  smoothing of the raw data. One of the key goals of statistics is to help
  compress the variation in a complex system into a few easily understandable
  components, and there is no reason why these response surfaces need to be
  thought of differently than the curves / surfaces learned by, say, smoothing
  splines or PCA, for example, except perhaps they may be somewhat more
  complicated to understand directly.
\item Characterizing Uncertainty: The test error in a machine learning algorithm
  gives a sense of how powerful the theory is / how much intrinsic noise there
  is the system under study.
\end{itemize}

\section{Methods}

\subsection{Features}

\subsection{Interpretation}

\subsection{Infrastructure}

\section{Case Studies}

\subsection{Cleanout Perturbation Study}

\bibliographystyle{unsrt}
\bibliography{bibliography}

\end{document}
